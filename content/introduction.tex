%----------------------------------------------------------------------------
\section{\bevezetes}
%----------------------------------------------------------------------------

%--------------------------------
% advancement in ML, GPUs, big data

Az elmúlt években a nagyméretű adathalmazok elérhetősége és a számítási kapacitás exponenciális növekedésének köszönhetően jelentősen megnövekedett az érdeklődés a gépi tanulás iránt. Manapság a gépi tanulási algoritmusokat előszeretettel alkalmazzák nagy dimenziós bemeneti adatokhoz kapcsolódó osztályozási, regressziós, klaszterezési vagy dimenziócsökkentési feladatokra. A gépi tanulási algoritmusok számos területen embert meghaladó képességekkel rendelkeznek (például képosztályozásban). A mindennapi életünkben használt okos eszközökön a kép- és beszédfelismerést, az internetes keresést, az arcfelismeréses bejelentkezést és még sok mást a gépi tanulási algoritmusok tesznek lehetővé.

% In recent years, the availability of large datasets combined with the improvement in algorithms and the exponential growth in computing power led to an unparalleled surge of interest in the topic of machine learning. Nowadays, machine learning algorithms are successfully employed for classification, regression, clustering, or dimensionality reduction tasks of large sets of especially high-dimensional input data.1 In fact, machine learning has proved to have superhuman abilities in numerous fields (such as playing go,2 self driving cars,3 image classification,4 etc). As a result, huge parts of our daily life, for example, image and speech recognition,5,6 web-searches,7 fraud detection,8 email/spam filtering,9 credit scores,10 and many more are powered by machine learning algorithms.

A gépi tanulás nagy részben hozzájárult a tudomány és technológia fejlődéshez. Szinte minden iparág és vállalat felismerte a gépi tanulás előnyeit és lehetőségeit. A technológia fejlődése lehetővé tette a közelmúltbeli áttöréseket, amelyek elősegítik a gyorsabb és hatékonyabb üzleti intelligenciát, az arcfelismeréstől a természetes nyelvfeldolgozásig terjedő alkalmazások felhasználásával.

% Advances in this technology have allowed for recent breakthroughs that promote faster and more efficient business intelligence, using abilities ranging from facial recognition to natural language processing.

% --------------------------------
% adatgeneráló eljárások megjelenése, motivációja (moar data)

A gépi tanulás alkalmazásai között szerepelnek olyan eljárások, amelyek képesek adatot generálni a bemenetből, mint például a generatív modellek vagy a deep metric learning, de ide sorolhatóak a napjainkban egyre elterjedtebben használt gépi tanulásra épülő arcfelismerési rendszerek is.

A generatív modellek, mint a GAN (Generative Adversarial Network) olyan neurális hálózatok, amelyek képesek új, szintetikus adatot előállítani. A háló bemenetén megadott véletlenszerű zaj hatására realisztikus képeket tudnak generálni. A GAN-ok elsősorban olyan szituációkban lehetnek nagyon hasznosak, amikor több tanítóadatra van szükségünk, viszont az adat gyűjtése nehezen megoldható. Sok esetben az adatgyűjtés és címkézés hosszadalmas, drága folyamat. Fontos viszont, hogy a generált szintetikus adat minősége megfelelő legyen a kívánt célra. Az új adatoknak megfelelően realisztikusnak kell lenniük ahhoz, hogy a generált adatokból szerzett ismeretek továbbra is érvényesek legyenek a valós adatokra. A generatív modellek leírják egy adathalmaz előállításának módját a valószínűségi modell alapján. Ebből a modellből mintavételezéssel új adatokat tudunk előállítani. A GAN -ok olyan struktúrát fedeznek fel az adatokban, amely lehetővé teszi számukra, hogy reális adatokat állítsanak elő.

% GAN is a type of neural network that is able to generate new data from scratch. You can feed it a little bit of random noise as input, and it can produce realistic images of bedrooms, or birds, or whatever it is trained to generate. One thing all scientists can agree on is that we need more data. GANs, which can be used to produce new data in data-limited situations, can prove to be really useful. Data can sometimes be difficult and expensive and time-consuming to generate. To be useful, though, the new data has to be realistic enough that whatever insights we obtain from the generated data still applies to real data. If you’re training a cat to hunt mice, and you’re using fake mice, you’d better make sure that the fake mice actually look like mice.


% A generative model can be broadly defined as follows: A generative model describes how a dataset is generated, in terms of a probabilistic model. By sampling from this model, we are able to generate new data. Another way of thinking about it is the GANs are discovering structure in the data that allows them to make realistic data. This can be useful if we can’t see that structure on our own or can’t pull it out with other methods. 

% A mélygenerációs modellek (DGM) alkalmazásai, mint például a híres képekből hamis portrék készítése, nemrégiben címlapokra kerültek. Ezen úgynevezett mély hamisítványok megjelenése jelentős társadalmi és jogi kihívások elé állít, de új előnyös technológiákat is ígér.

% Applications of deep generative models (DGM), such as creating fake portraits from celebrity images, have recently made headlines. The advent of these so-called deep fakes poses considerable societal and legal challenges, but also promise new beneficial technologies [10].

% --------------------------------
% arcfelismerés -> egyre több kamera

A gépi tanulási technikák fejlődésének köszönhetően, illetve az egyre olcsóbb okos eszközök elterjedésével egyre jobban elterjedt az arcfelismerés használata. A generatív modellekhez hasonlóan, az modern arcfelismerő rendszerek is gépi tanulásra támaszkodnak. Működésük során az emberek arcáról készült digitális képekből képesek arcot jellemző metrikákat előállítani oly módon, hogy egy emberhez tartozó arcleíró vektorok (későbbiekben arclenyomatok) távolsága kicsi legyen, míg különböző emberek között nagy. Erre a célra létrehozott neurális hálók képesek megtanulni azt a deep metrikát ami legjobban leírja az emberi arc struktúráját.

% arcfelismerés és adatvédelem -> arclenyomatvektor

Az ilyen módon kinyert arclenyomatokban tárolt információ ember számára nem értelmezhető, de belőlük gépi tanulási módszerekkel személyes adat származtatható a képen látható személyekről. Magukba kódolva olyan információkat is hordoznak, mint például a rassz, a nem és az arc egyéb jellemzői. Ezek az információk együttesen lehetővé teszik egy személy azonosítását.

% While these seem as a list of arbitrary numbers to the naked eye, they may contain personal information about the person whose photo was taken. In their recent work, Mai et al. showed that the photo itself can be reconstructible from the embedding. In authors argue that it should be an accepted fact that with good accuracy the original sample can be reconstructed from unprotected embeddings. This means that sensitive data could be derived from unprotected templates and other attacks can also be launched based on the reconstruction results. Based on this, it can also be possible to reverse engineer data from face embeddings in order to find out the original identity of the embedding

% cél

A dolgozatomban azt vizsgálom, hogy a gépi tanulás alapú arcfelismerés által kinyert arclenyomatok lehetővé teszik-e az embedding forrását, az eredeti alanynak a felismerését. Megvizsgálom, hogy az arclenyomatok hogyan kódolják az arc struktúráját, illetve, hogy annak mely részei hordoznak érzékeny információt az adatalanyról. Célom arclenyomatok vizsgálatát olyan módszerrel végezni, amely általánosítható lehet akár generatív modelleknél használt belső reprezentációk analízisére.

A dolgozatom felépítése a következő. A \ref{sec:irodalom}. fejezetben a témámhoz kapcsolódó fontosabb témaköröket mutatom be, a 3.[REF] fejezetben az arclenyomatokhoz kapcsolódó adatvédelmi kérdésekkel foglalkozom. A 6.[REF] fejezetben azt vizsgálom, milyen információkat hordozhat egy embedding, azok miként manipulálhatóak. a 7.[REF] fejezetben javaslatot teszek az adatvédelmi kockázatok kezelésére.

% Ezért munkám során az elmúlt időszakban azt vizsgáltam, hogy a gépi tanulás alapú arcfelismerés által kinyert ún. embeddingek, lehetővé teszik-e az embedding forrását, az eredeti adatalanynak a felismerését, és ha igen, akkor mely részei kódolják az egyes információkat. A célom, hogy meghatározzam az arclenyomat vektorok (embeddingek) azon koordinátáit, amelyek az érzékeny információkat hordozzák, például a rasszt. Ezt különböző gépi tanulás elemzésére és értelmezésére használható eszközökkel hajtom végre. Az itt meghatározott koordinátákat módosítjuk vagy töröljük. Ezek után vizsgáljuk, hogy ennek hatására hogyan változik az érzékeny információ meghatározására alkalmas modellek pontossága.

% We live in times when efficient uses of artificial intelligence and cheap smart technology are exploding. By the spread of smart cameras, applications on facial recognition had become almost ubiquitous in some cities around the world. In some cases we can find the driver reason for this in the security concerns of the public, but face recognition (or FR in short) can be applied to a much broader set of use-cases. Beside identification or authentication of individuals in crowds, it could benefit the society also in criminal detection, searching for lost people, customer behavior analysis, etc.

% However, FR technology could be abused and therefore it has the potential to pose risks to individuals, to the society and even to the governmental and business sectors, as well. This puts related ethical issues into the focus.

% The 2010s kickstarted the modern era of facial recognition, as computers were finally powerful enough to train the neural networks required to make facial recognition a standard feature.
% Facial recognition first trickled into personal devices as a security feature with Windows Hello and Android’s Trusted Face in 2015, and then with the introduction of the iPhone X and Face ID in 2017.

% Proponents of facial recognition suggest that the software is useful because alongside identifying suspects, it can monitor known criminals and help identify child victims of abuse. In crowds, it could monitor for suspects at large events and increase security at airports or border crossings. The most long-running type of facial recognition software runs a photo through a government-controlled database, such as the FBI’s database of over 400 million photos, which includes driver’s licenses from some states, to identify a suspect. Local police departments use a variety of facial recognition software, often purchased from private companies.

% There’s a long list of benefits facial recognition can offer outside of law enforcement, adding convenience or security to everyday things and experiences. Facial recognition is helpful for organizing photos, useful in securing devices like laptops and phones, and beneficial in assisting blind and low-vision communities. It can be a more secure option for entry into places of business, fraud protection at ATMs, event registration, or logging in to online accounts. Advertising and commercial applications of facial recognition promise a wide array of supposed benefits, including tracking customer behavior in a store to personalize ads online.

% Face recognition systems are being increasingly used for secure access in applications ranging from personal devices to access control (e.g., banking and border control). In critical applications, face recognition needs to meet stringent performance requirements, including low error rates and strong system security. In particular, the face recognition system must be resistant to spoofing (presentation) attacks and template invertivility. Therefore, it is critical to evaluate the vulnerabilities of a face recognition system to these attacks and devise necessary countermeasures. To this end, several attack mechanisms (such as hill climbing [1], [2], [3], spoofing [4], [5], [6], [7], [8], and template reconstruction (template invertibility) [9]) have been proposed to determine the vulnerabilities of face recognition systems.

% A 2. chapterben ... a 3. chapterben ... a 4. chapterben ...