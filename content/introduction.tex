%----------------------------------------------------------------------------
\chapter{\bevezetes}
%----------------------------------------------------------------------------

Gépi tanulásos rendszereket egyre szélesebb körben alkalmaznak komplex feladatok megoldására, többek között az orvostudományban, pénzügyi technológiákban illetve az iparban is. A gépi tanulás nagy részben hozzájárult a tudomány és technológia fejlődéshez. Szinte minden iparág és vállalat felismerte a gépi tanulás előnyeit és lehetőségeit.







%----------------------------------------------------------------------------
\newpage
Felépítés:
\begin{itemize}
	\item gépi tanulás jelentősége \\ napjainkban egyre elterjedtebb, hol használják
	\item adatgeneráló eljárások \\ ... ennek utánajár
	\item arcfelismerés bemutat: egyre gyakrabban használnak kamerát \\ térfigyelő kamerák, ...
	\item adatvédelmi kockázat
	\begin{itemize}
		\item pl kína szociális kreditrendszer
		\item  arcfelismerő rendszerek tökéletlenségét kihasználó hozzáférési támadások (pl. "face morphing", "presentation attacks", stb.)
		\item a gépi tanulás hibáiból bekövetkező diszkriminációk és pontatlanságok
		\item az arclenyomatokból gépi tanulási eljárásokkal kiszedhető személyes információk kiszivárgása
	\end{itemize}
	\item motiváció, mit vizsgálok a dolgozatban
	\item diplomaterv felépítése
\end{itemize}
