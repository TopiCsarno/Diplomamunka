%----------------------------------------------------------------------------
\section{Bevezetés (2-3 oldal)}
%----------------------------------------------------------------------------

Advances in machine learning ...

Felépítés:
\begin{itemize}
	\item gépi tanulás advancements
	\item gépi tanulás jelentősége \\ napjainkban egyre elterjedtebb, hol használják
	\item deep learning elterjedése, használata
	\item adatgeneráló eljárások \\ motiváció, fajták, felhasználás
	\item embeddingek, face embedding
	\item arcfelismerés bemutat: egyre gyakrabban használnak kamerát \\ térfigyelő kamerák, ...
	\item valami szemléltető ábra hasznos lenne, működés magyarázatához. ábra: emberek -> kamera -> képet készít -> embedding -> tárolás, feldolgozás
	\item adatvédelmi kockázat
	\begin{itemize}
		\item pl kína szociális kreditrendszer
		\item  arcfelismerő rendszerek tökéletlenségét kihasználó hozzáférési támadások (pl. "face morphing", "presentation attacks", stb.)
		\item a gépi tanulás hibáiból bekövetkező diszkriminációk és pontatlanságok
		\item az arclenyomatokból gépi tanulási eljárásokkal kiszedhető személyes információk kiszivárgása
	\end{itemize}
	\item motiváció, mit vizsgálok a dolgozatban
	\item diplomaterv felépítése
\end{itemize}

%----------------------------------------------------------------------------
\section{Irodalomkutatás (15-20 oldal)}
%----------------------------------------------------------------------------
Átfogó jelleggel mutassa be az adat generáló gépi tanulási megoldásokat!

Az egészet időrendi sorrendben kéne felvezetni -> régi: Boltzman -> új \\
Fontos része: mögöttes ötlet -> embedding / látens térbeli reprezentáció

Syamese network -> ez is embeddingeket hoz létre.

Probléma felvetés:
Ötlet: általánosítható legyen a módszer akár többféle alkalmazásban is. -> Deep fake reprezentáció is tartalmazhat személyes adatot

Felépítés: \\
(4-5 topic ~3-4 oldal)
\begin{itemize}
	\item felvezető ~ 1 oldal 
	\begin{itemize}
		\item history -> machine learning honnan jött
		\item deep learning, gpu, data
		\item motiváció
		\item synthetic data
		\item training data előállítása
		\item unsupervised learning
	\end{itemize}
	\item boltzman machines (~2-3)
	\item generatív modellek működése, példákkal (GAN) (~3-4)
	\item variational auto-encoder (~2-3)
	\item deep metric learning -> embedding (~1-2)
	\item syamese networks (~2 oldal)
\end{itemize}

%----------------------------------------------------------------------------
\section{Probléma bemutatása (4-6 oldal)} 
%----------------------------------------------------------------------------
\begin{itemize}
	\item arcfelismerő rendszerek működése
	\item arcfelismerés problémakör
	\item arclenyomat vektor
\end{itemize}

%----------------------------------------------------------------------------
\section{Arclenyomat vektorok adatvédelmi elemzése (8-10) oldal}
%----------------------------------------------------------------------------
\begin{itemize}
	\item Módszertan, támadó modell 
	\item Használt adathalmaz előállítása (korábbi munka:
	VGG dataset) 
	\item Predikciós modellek tanítása: sex, race, age 
	\item Újraazonosíthatóság kockázata
\end{itemize}

%----------------------------------------------------------------------------
\section{Az embeddingben kódolt személyes adatok vizsgálata (10-12 oldal)}
%----------------------------------------------------------------------------
\begin{itemize}
	\item Legfontosabb embedding feature-ök meghatározása (kutatási jelentésből)
	\item Legfontosabb feature-ök eltávolítása, módosítása milyen hatással van a modell pontosságára (kutatási jelentésből) 
	\item Eredmények kiértékelése (kutatási jelentésből) 
	\item Arclenyomatvektor feature-einek köncsönös összefüggésének vizsgálata (diploma A-ból)
\end{itemize}


%----------------------------------------------------------------------------
\section{Javaslat a kockázatok kiszűrésére (10+ oldal)}
%----------------------------------------------------------------------------
\begin{itemize}
	\item Döntési fák félrevezetése embeddingek kis módosításával (diploma A-ból)
	\item TODO
\end{itemize}

%----------------------------------------------------------------------------
\section{Összefoglalás (1-3 oldal)}
%----------------------------------------------------------------------------
