% %----------------------------------------------------------------------------
% \section{Bevezetés (2-3 oldal)}
% %----------------------------------------------------------------------------

% Advances in machine learning ...

% Felépítés:
% \begin{itemize}
% 	\item gépi tanulás advancements
% 	\item gépi tanulás jelentősége \\ napjainkban egyre elterjedtebb, hol használják
% 	\item deep learning elterjedése, használata
% 	\item adatgeneráló eljárások \\ motiváció, fajták, felhasználás
% 	\item embeddingek, face embedding
% 	\item arcfelismerés bemutat: egyre gyakrabban használnak kamerát \\ térfigyelő kamerák, ...
% 	\item valami szemléltető ábra hasznos lenne, működés magyarázatához. ábra: emberek -> kamera -> képet készít -> embedding -> tárolás, feldolgozás
% 	\item adatvédelmi kockázat
% 	\begin{itemize}
% 		\item pl kína szociális kreditrendszer
% 		\item  arcfelismerő rendszerek tökéletlenségét kihasználó hozzáférési támadások (pl. "face morphing", "presentation attacks", stb.)
% 		\item a gépi tanulás hibáiból bekövetkező diszkriminációk és pontatlanságok
% 		\item az arclenyomatokból gépi tanulási eljárásokkal kiszedhető személyes információk kiszivárgása
% 	\end{itemize}
% 	\item motiváció, mit vizsgálok a dolgozatban
% 	\item diplomaterv felépítése
% \end{itemize}

% %----------------------------------------------------------------------------
% \section{Irodalomkutatás (15-20 oldal)}
% %----------------------------------------------------------------------------
% Átfogó jelleggel mutassa be az adat generáló gépi tanulási megoldásokat!

% Az egészet időrendi sorrendben kéne felvezetni -> régi: Boltzman -> új \\
% Fontos része: mögöttes ötlet -> embedding / látens térbeli reprezentáció

% Syamese network -> ez is embeddingeket hoz létre.

% Probléma felvetés:
% Ötlet: általánosítható legyen a módszer akár többféle alkalmazásban is. -> Deep fake reprezentáció is tartalmazhat személyes adatot

% Felépítés: \\
% Adatgeneráló eljárások (4-5 topic ~3-4 oldal)
% \begin{itemize}
% 	\item felvezető ~ 4 oldal 
% 	\begin{itemize}
% 		\item supervised / unsupervised learning (mygreatlearning GAN)
% 		\item discriminate / generative models
% 		\item density estimation (mygreatlearning Boltz)
% 		\item generative modellek csoportosítása (Explicit, implicit) (mygreatlearning Boltz)
% 	\end{itemize}
% 	\item boltzman machines (~3)
% 	\begin{itemize}
% 		\item extract latent space from data
% 		\item Markov chain
% 		\item graphical models -> remove mb
% 		\item Markov property
% 		\item BM and RBM structure - how it works
% 		\item Usage today (spoiler alert: not much)
% 		\item ha kell több emlélet: deeplearningbook.org
% 	\end{itemize}
% 	\item Generative adversarial network (GAN) (~3-4)
% 	\begin{itemize}
% 		\item mygreatlearning explanation
% 		\item input noise generates the new data
% 		\item Generator and Discriminator
% 		\item they compete and get better -> discriminator cant tell anymore
% 		\item GAN példák -> TODO
% 	\end{itemize}
% 	\item variational auto-encoder (~2-3)
% 	\begin{itemize}
% 		\item mygreatlearning explanation
% 		\item auto encoder úgy általában. (dim(input) = dim(output))
% 		\item compress to small representation -> uncompress
% 		\item encoder, 'code', decoder
% 		\item variational auto encoder mitől "variational"
% 		\item data follows normal distribution -> mean, std
% 		\item how it's trained (reparameterization trick, loss function)
% 		\item példa alkalmazásra: deep fake -> TODO
% 	\end{itemize}

% 	\item deep metric learning -> embedding (~1-2)
% 	\begin{itemize}
% 		\item 
% 	\end{itemize}
% 	\item syamese networks (~2 oldal)
% 	\begin{itemize}
% 		\item 
% 	\end{itemize}
% 	\item triplet loss
% 	\begin{itemize}
% 		\item 
% 	\end{itemize}
% \end{itemize}

%----------------------------------------------------------------------------
% \section{Probléma bemutatása (4-6 oldal)} 
% %----------------------------------------------------------------------------
% \begin{itemize}
% 	\item arcfelismerő rendszerek működése
% 	\item arcfelismerés problémakör
% 	\item arclenyomat vektor
% \end{itemize}

%----------------------------------------------------------------------------
\section{Arclenyomatok adatvédelmi elemzése (8-10) oldal}
%----------------------------------------------------------------------------
\begin{itemize}
	\item módszertan, támadómodell -> hozzáfér datasethez
	\item adatgyűjtés -> vgg és imdb datasetek előállítása (2-4 oldal)
	\item random forestek betanítása -> race, sex, age, +1 valami?
	\item betanítás utáni eredmények -> accuracy, ROC curve
	\item adatvédelmi elemzés -> milyen riskek vannak (munkacsoport szerint)
	\item pl: több info kiderülése 1 emberről probléma, adatszivárgás: le tudja szűkíteni, hogy ki lehet az, singling out, vagy lehet arcot rekonstruálni az embeddingből, újraazonosítás.
\end{itemize}

%----------------------------------------------------------------------------
\section{Az embeddingben kódolt személyes adatok vizsgálata (10-12 oldal)}
%----------------------------------------------------------------------------
\begin{itemize}
	\item Legfontosabb embedding feature-ök meghatározása (kutatási jelentésből)
	\item Legfontosabb feature-ök eltávolítása, módosítása milyen hatással van a modell pontosságára (kutatási jelentésből) 
	\item Eredmények kiértékelése (kutatási jelentésből) 
	\item Arclenyomatvektor feature-einek köncsönös összefüggésének vizsgálata (diploma A-ból)
\end{itemize}


%----------------------------------------------------------------------------
\section{Javaslat a kockázatok kiszűrésére (10+ oldal)}
%----------------------------------------------------------------------------
\begin{itemize}
	\item Döntési fák félrevezetése embeddingek kis módosításával (diploma A-ból)
	\item Adversarial attack módszerek körbejárása
	\item javaslat: titkosítás, lsh, random projekció (source: MNB arcfelismerés)
\end{itemize}

%----------------------------------------------------------------------------
\section{Összefoglalás (1-3 oldal)}
%----------------------------------------------------------------------------
